\section{Theorems}

In this section I will present the theorems that I proved for formal languages using Stainless, and the sketch of their implementation. Note that this enumeration is not an exhaustive list, some helper lemmas might not be presented. In some cases, theorem proofs are omitted because their simplicity or analogousness to other proofs, or because they would require too much space. For the list and proofs of all theorems, see the source code, available on \href{https://github.com/czipobence/semesterProject-2017-autumn-assignments/tree/master/FormalLanguages}{GitHub}.

\subsection{Extending the implementation}

In this section, all theorems and stainless proofs are phrased and presented considering an implementation with not unique and not ordered Lists. A sketch for such implementation has been presented in Section~\ref{set:implList}, however it misses some functionality. This section details the methods for this implementation and introduces notations and abbreviations used in this report.

\subsubsection{Equalities}

In the sketch, we used the \inline{==} operator to check if two languages are the same.However, overriding such operator can cause unexpected behavior in Stainless, and for this reason, we need to define our own operator for that.

\begin{ShortCode}{Stainless}
 def sameAs(that: Lang[T]): Boolean = {
   this.list.content == that.list.content
 }
\end{ShortCode}

\subsubsection{Set operations}

The sketch presented the union operation \inline{++} and the containment operation \inline{contains}. For some cases, we might like to add an other single word to the language, which can be done with the \inline{::} operator.

\begin{ShortCode}{Stainless}
 def ::(t: List[T]): Lang[T] = Lang[T](t :: this.list)
\end{ShortCode}

An other convenient is inclusion, to check if a language is contained in an other. 

\begin{ShortCode}{Stainless}
 def subsetOf(that: Lang[T]): Boolean = 
                  this.list.content subsetOf that.list.content
\end{ShortCode}

Finally, it could be also benefitial to not only allow adding new elements, but also allow removing elements from the language. For this reason lets introduce the \inline{--} operator.

\begin{ShortCode}{Stainless}
def --(that: Lang[T]): Boolean = Lang[T](this.list--that.list)
\end{ShortCode}

\subsubsection{Language operations}

The implementation for concatenation of languages has already been presented. However, for efficiency, we can define more operations.

The power of a language can be defined the following way:

\begin{ShortCode}{Stainless}
 def ^ (i: BigInt): Lang[T] = i match {
   case BigInt(0) => Lang[T](List(Nil()))
   case _ => this concat (this ^ (i-1))
 }
\end{ShortCode}

Note that this operation unfolds to the right. We could have defined it in an other way, as 

\begin{ShortCode}{Stainless}
 def :^ (i: BigInt): Lang[T] = i match {
   case BigInt(0) => Lang[T](List(Nil()))
   case _ =>  (this ^ (i-1)) concat this
 }
\end{ShortCode}

We will later show that these two operations are equivalent but until then, lets keep both definitions.

Having defined the power of a language, we can take a step towards having the star of a language by defining the n-close of a language. First lets define it formally the following way: 

\begin{definition}[Close of a language]
	Let $\languagei{}$ be a language. The n'th close of the language can be defined as $ \languageclose{n} = \bigcup_{i=0}^n \languagepow{i}$
\end{definition}

This definition has to really useful properties:

\begin{itemize}
	\item if $\wordsymbol \in \languageclose{n}$ then $\forall n' \ge n. ~ \wordsymbol \in \text{close}_{n'}(\languagei{})$ and $\wordsymbol \in \languagepow{*}$
	\item if $\wordsymbol \in \languagepow{*}$ then $\exists n. ~ \wordsymbol \in \languageclose{n}$ 
\end{itemize}

In other words, if we would like to prove that a word is in the star of a language, it is enough to construct an integer n such that the word will be the n'th close of the language. The big advantage compared to the power of languages is that in this case we only have to tell an upper bound for $j$ for which the word is in $\languagepow{j}$.

Such function can be implemented the following way:

\begin{ShortCode}{Stainless}
 def close(i: BigInt): Lang[T] = i match {
   case BigInt(0) => this ^ i
   case _ => (this close (i-1)) ++ (this ^ i)
 }
\end{ShortCode}

\subsubsection{Helper methods}

In order to be able to efficiently define the language operations, some helper methods have been already introduced. Namely \inline{concatList} which in functionality is basically the same as the concat operation, just it is for lists not languages. The \emph{appendToAll} function takes a word and a language and appends the word to the end of each word in that language.

Intuitively, as we defined \inline{appendToAll}, we might want to define a similar operation \inline{prependToAll}, which prepends a single word to any other words in a language.

\begin{ShortCode}{Stainless}
 def prependToAll[T](prefix:List[T], 
                     l:List[List[T]]):List[List[T]] = l match {
   case Nil() => Nil[List[T]]()
   case hd::tl => (prefix ++ hd) :: prependToAll(prefix, tl) 
 }
\end{ShortCode}




\subsection{Notations}

Writing long proofs using operation names and strict syntax can be space demanding. For sake of clearness and easier understanding, lets introduce some notations.

From now on, lets use \inline{==} for language equality, and \inline{===} for strict equality, namely marking the equality of the representing lists.

Instead of \inline{++} and \inline{--} operators, use the corresponding set operations, $\cup$ and $\setminus$.

Also let $\wordsymbol \in L$ denote the \inline{contains} function, and use $\subseteq$ as a shorthand for \inline{subsetOf}. For concatenation, we can use the $\concatop$ operator, or to make it more function-like, we can use $cl(\languagei{1} , \languagei{2})$. Since \emph{concat} and \emph{concatLists} are functionally really similar, we can use the same notation for both. Lets also abbreviate the function names, \inline{appendToAll} as aa and \inline{prependToAll} as pa.

We can also define notations for the types. We can get rid of the type parameter T, since all operations are only valid on languages and words over the same alphabet. Let \inline{Word} denote \inline{List[T]} and let \inline{Lang} denote \inline{Lang[T]}. For the sake of simplicity lets also suppose that all list operations can be invoked on languages, without accessing the underlying list (so for a language \inline{l} and a list method \inline{f(...)} let \inline{l.f(...)} denote \inline{l.list.f(...)}).

Also for notation, let $\emptyset$ denote the empty language, which is with this representation is equivalent to \inline{Nil[List[T]]()}. The empty word with this representation would be \inline{Nil[T]()} so the unit language $\unitlang$ is \inline{List(Nil[T]())}

\subsection{Theorems about words}

Words are list of symbols, thus many theorems about them hold, because they hold for any kind of list. It is not easy to see that since a symbol can be literally anything, every theorem that holds for a word constructed from symbols must hold for lists of any type.

\subsubsection{Empty word concatenation}

We can state a theorem that the concatenation of an empty word is an identity operation.

\begin{theorem}
	\label{the:emptyword}
	Let $\wordsymbol$ be a word. In this case 
	\begin{itemize}
		\item $\wordsymbol \concatop \emptyword = \wordsymbol$
		\item $\emptyword \concatop \wordsymbol = \wordsymbol$
	\end{itemize}
\end{theorem}

We can also state some kind of associativity for word concatenation:

\begin{theorem}
	Let $\wordi{1}, \wordi{2}, \wordi{3}$ be words over the same alphabet $\alphabetSymbol$. In that case, $(\wordi{1} \concatop \wordi{2}) \concatop \wordi{3} = \wordi{1} \concatop (\wordi{2} \concatop \wordi{3})$ 
\end{theorem}

The corresponding theorems and proofs for all the above are implemented in Stainless, in \inline{stainless.collection.ListSpecs}.

\begin{theorem}[Cancellation Laws]
	For words $\wordi{1}, \wordi{2}, \wordi{3}$ over the same alphabet $\alphabetSymbol$, we have
	\begin{itemize}
		\item if $\wordi{1} \concatop \wordi{2} = \wordi{1} \concatop \wordi{3}$ then $\wordi{2} = \wordi{3}$
		\item if $\wordi{2} \concatop \wordi{1} = \wordi{3} \concatop \wordi{1}$ then $\wordi{2} = \wordi{3}$
	\end{itemize}
\end{theorem}

\subsection{Theorems About Languages}

As it was mentioned in Section~\ref{sect:impl.comp}, I used list of words to represent a language, so whenever a theorem is phrased, we actually phrase it of a list of list of T, or more precisely, since I chose the non-unique variant, about the content of such lists of lists.


\subsubsection{Combination to the Empty Language}

\begin{theorem}[Null Combination]
	Any language concatenated to the empty language results in the empty language, formally,
	\begin{itemize}
		\item $\forall \languagei{}. ~ \languagei{} \concatop \emptyset = \emptyset$
		\item $\forall \languagei{}. ~ \emptyset \concatop \languagei{} = \emptyset$
	\end{itemize}
\end{theorem}

Converting this theorems into stainless format is straightforward, however, as a reference, this time it is included.

Actually, we can convert them  differently:

\begin{ShortCode}{Stainless}
 def rightNullConcat[T](l: Lang[T]): Boolean = {
   l.concat(nullLang[T]()) sameAs nullLang[T]()
 }.holds	
 
 or
 
 def rightNullConcat[T](l1: Lang[T]): Boolean = {
   forall( (l: Lang[T]) => 
       l.concat(nullLang[T]()) sameAs nullLang[T]()
   )
 }.holds
 
 or directly on lists of lists
 
 def rightNullConcat[T](l: List[List[T]]): Boolean = {
    concatLists(l, Nil[List[T]]()).content == 
 	                               Nil[List[T]]().content
 }.holds
 
\end{ShortCode}

The difference between the first and the second is that when we will later reference a theorem (or lemma) in a proof of an other theorem, in the first case we can exactly state for which language should the solver apply the theorem, which enhances the performance. Comparing the first and the third, they are basically the same, and if Stainless can not prove a theorem, eventually we will have to give hints to the solver based on the underlying data structures, which in this case is a list.

Recall that in the implementation presented in Section~\ref{set:implList}, we applied structural induction on the right hand operand of the concatenation. For this reason the first case is straightforward, because the operation will yield an empty list. 

Note that in this case, we proved the equality of the lists, that are representing the languages, instead of proving the equality of their content. However it is straightforward that if two lists are identical, their content are the same.

Similarilly, we can state the second part of the theorem:

\begin{ShortCode}{Stainless}
 def leftNullConcat[T](l: Lang[T]): Boolean = {
   (nullLang[T]()).concat(l) sameAs nullLang[T]()
 }.holds	
\end{ShortCode}

Stainless can verify the theorem above without further aid, however we can also prove it by hand. If L is an empty list, the proof is just as straightforward as in the first case. On the other hand if it has some elements, we can apply induction.

\begin{ShortCode}{Stainless}
 cl($\emptyset$, hd::tl) ==
 aa($\emptyset$, hd) ++ cl($\emptyset$, tl) ==
 $\emptyset$ ++ cl($\emptylist$, tl) ==
 cl($\emptyset$, tl) == //by the induction hypothesis
 $\emptyset$
\end{ShortCode}

\subsubsection{Structural Theorems and Lemmas}

The previous theorems were straightforward from the definition, Stainless can prove them without a hint. However, in order to prove more complex theorems, we have to define some helper lemmas and theorems.

First of all, we would like to show, that even though we defined list concatenation with appending, we could have defined it with perpending, namely that 

\begin{lemma}
	\label{lem:clInductLeft}
	For any languages $\languagei{1}, \languagei{2}$ over the same alphabet, where $\languagei{1} = hd1::tl1$ we have
	cl(hd1::tl1, $ \languagei{2}$) == pa(hd1,$ \languagei{2}$) ++ cl(tl1,$ \languagei{2}$)
\end{lemma}

If $ \languagei{2}$ is $\emptyset$, we can apply the previous theorem and have \inline{cl((hd1::tl1), $\emptyset$) = $\emptyset$} and \inline{ [pa(hd,$\emptyset$)] = $\emptyset$  } and \inline{cl(tl,$\emptyset$) = $\emptyset$}, and the proof is straightforward.

Otherwise, we can apply induction. We know that \inline{$ \languagei{2}$ = hd2 :: tl2}, so we have

\begin{ShortCode}{Stainless}
 cl(hd1::tl1, hd2::tl2) === 
 //by definition
 aa(hd1::tl1, hd2) ++ cl(hd1::tl1, tl2) === 
 //by definition
 [hd1 ++ hd2] ++ aa(tl1, hd2) ++ cl(hd1::tl, tl2)  === 
 // by the I.H.
 [hd1 ++ hd2] ++ aa(tl1, hd2) ++ pa(hd1, tl2) ++ cl(tl1, tl2)
\end{ShortCode} 

Similarly, we can also show that 

\begin{ShortCode}{Stainless}
 [hd1++hd2] ++ pa(hd1, tl2) ++ aa(tl1, hd2) ++ cl(tl1, tl2) ===
 pa(hd1, hd2 :: tl2) ++ aa(tl1, hd2) ++ cl(tl1, tl2) ===
 pa(hd1, hd2 :: tl2) ++ cl(tl1, hd2 :: tl2)
\end{ShortCode} 

See that in the two claims above, we used strict equality instead of set equality, so we can apply that if for two languages $\languagei{1} === \languagei{2}$ then $\languagei{1} == \languagei{2}$.

So we can state that:

\begin{ShortCode}{Stainless}
 cl(hd1::tl1, hd2::tl2) ==
 [hd1++hd2] ++ aa(tl1, hd2) ++ pa(hd1, tl2) ++ cl(tl1, tl2) ==
 [hd1++hd2] ++ pa(hd1, tl2) ++  aa(tl1, hd2) ++ cl(tl1, tl2) ==
 pa(hd1, hd2 :: tl2) ++ cl(tl1, hd2 :: tl2)
\end{ShortCode}

The proof of this lemma can be found in the code in function \inline{clInductLeft}.

Intuitively, we can generalize the following lemma to some kind of distributivity.

\begin{lemma}[Left Distributivity]
	\label{lem:clLeftDistributiveAppend}
	For any languages $\languagei{1}, \languagei{2}, \languagei{3}$ we have $(\languagei{1} ++ \languagei{2}) \concatop \languagei{3} == \languagei{1} \concatop \languagei{3} ++ \languagei{2} \concatop \languagei{3}$.
\end{lemma}

\begin{lemma}[Right Distributivity]
	\label{lem:clRightDistributiveAppend}
	For any languages $\languagei{1}, \languagei{2}, \languagei{3}$ we have $(\languagei{1} ++ \languagei{2}) \concatop \languagei{3} == \languagei{1} \concatop \languagei{3} ++ \languagei{2} \concatop \languagei{3}$.
\end{lemma}

The proof for these two lemmas are not included, but in can be seen that they can be devised inductively. They are implemented in functions \inline{clLeftDistributiveAppend} and \inline{clRightDistributiveAppend}.

\subsubsection{Concatenation to the Unit Language}

\begin{theorem}[Unit Concatenation (Right)]
	\label{the:rightUnitConcat}
	$\forall \languagei{} \subseteq \alphabetpow{*}.$ we have  $\languagei{} \concatop \unitlang = \languagei{}$
%$\unitlang \concatop \languagei{} = \languagei{}$
\end{theorem}

If L is an empty language, the solution is straightforward, we can apply the previous theorem, $\emptyset \concatop \{\emptyword\} = \emptyset$. Otherwise, we can apply structural induction on \inline{L = hd :: tl} the following way:

\begin{ShortCode}{Stainless}
 cl(hd::tl, $\unitlang$)              == // LM1
 pa(hd, $\unitlang$) ++ cl(tl, $\unitlang$)      == // LM2
 [hd] ++ cl(tl, $\unitlang$)          == // by the I.H.
 [hd] ++ tl                ==
 L 
\end{ShortCode}

Note that the step marked with LM1 applies Lemma~\ref{lem:clInductLeft}. The LM2 step is almost trivial to see, however Stainless can't prove it on its own. For the complete proof see the function \inline{prependToEmptyList}. The proof of the theorem can be found in function \inline{rightUnitConcat}.

\begin{theorem}[Unit Concatenation (Left)]
	\label{the:leftUnitConcat}
	$\forall \languagei{} \subseteq \alphabetpow{*}.$ we have $\unitlang \concatop \languagei{} = \languagei{}$
\end{theorem}


For this theorem, the solution is even easier. If L is an empty language, the solution is also straightforward, we have, $\{\emptyword\} \concatop \emptyset = \emptyset$.

If not, we know that \inline{L == hd :: tl} so we can write 

\begin{ShortCode}{Stainless}
 cl($\unitlist$, hd::tl) == 
 aa($\unitlist$, hd) ++ cl($\unitlist$, tl) == // Theorem~\ref{the:emptyword}
 [hd] ++ cl($\unitlist$, tl) == //by the I.H.
 [hd] ++ tl ==
 L
\end{ShortCode}

The implementation for this theorem is proved in the function \inline{leftUnitConcat}.

\subsubsection{Operations on Equivalent Languages}

Since different lists can represent the same language, it is not straightforward that applying the same operation on the same languages leads to the same result. We can, however, state some lemmas that will be useful in the next sections.

First of all, define that if two languages are the same, concatenating them to the same language yields the same language.

\begin{lemma}
	\label{lem:clContentEquals}
	Let $\languagei{1}, \languagei{2}, \languagei{3}$ be three languages in the same alphabet, and let $\languagei{1} == \languagei{2}$. Then $\languagei{1} \concatop \languagei{3} == \languagei{2} \concatop \languagei{3}$ 
\end{lemma}

We can also state the lemma for the other case, where the left hand side operator is fixed.

\begin{lemma}
	\label{lem:clContentEquals2}
	Let $\languagei{1}, \languagei{2}, \languagei{3}$ be three languages in the same alphabet, and let $\languagei{2} == \languagei{3}$. Then $\languagei{1} \concatop \languagei{1} == \languagei{1} \concatop \languagei{3}$ 
\end{lemma}

The proof for these lemmas are not included, but shown in functions \inline{clContentEquals} and \inline{clContentEquals2} and in other, referenced sub-lemmas.


\subsubsection{Associativity}

We have seen that for every $\languagesymbol \in \alphabetpow{*}$, $\languagesymbol \concatop \unitlang = \unitlang \concatop \languagesymbol = \languagesymbol$. Now the question arises if $(\alphabetpow{*};\concatop)$ is a monoid. For this, we have to prove associativity of languages over the concatenation.

Before that, lets define a helper lemma that will help us to prove associativity. 

\begin{lemma}
	\label{lem:replaceConcatPrepend}
	For any word $\wordsymbol$ and languages $\languagei{1}$ and $\languagei{2}$ we have $pa(\wordsymbol, \languagei{1}) \concatop \languagei{2} == pa(\wordsymbol, \languagei{1} \concatop \languagei{2})$
\end{lemma}

The lemma can be proved applying structural induction, and a proof for it can be found in the code under the function \inline{replaceConcatPrepend}.

\begin{theorem}[Associativity]
	\label{the:associativity}
	For every $\languagei{1}, \languagei{2}, \languagei{3} \subseteq \alphabetpow{*}$ we have $(\languagei{1} \concatop \languagei{2}) \concatop \languagei{3}  = \languagei{1} \concatop ( \languagei{2} \concatop \languagei{3} )$
\end{theorem}

A convenient proof for the theorem would be to show that each element in one language is contained in the other language. So if there exists an injection from one set to the other, and the other set to the first, then the two are identical. However, Stainless is more efficient when something is proved with structural induction so we will show associativity with structural induction.

We will apply induction on $\languagei{1}$, first lets examine the case when it is $\emptyset$. In this case on the left hand side of the equality, we have $(\emptyset \concatop \languagei{2}) \concatop \languagei{3} = \emptyset \concatop \languagei{3} = \emptyset$ and on the right hand side we have $\emptyset \concatop ( \languagei{2} \concatop \languagei{3} ) = \emptyset$, so the two are equal.

Otherwise, there is some hd and tl such that $\languagei{1} = hd :: tl$.

Now we can apply the following steps:

\begin{ShortCode}{Stainless}
 $((hd::tl) \concatop \languagei{2}) \concatop \languagei{3}$ == // (1)
 $(pa(hd, \languagei{2}) \cup (tl \concatop \languagei{2})) \concatop \languagei{3}$ == // (2)
 $(pa(hd, \languagei{2}) \concatop \languagei{3}) \cup ((tl \concatop \languagei{2}) \concatop \languagei{3})$ == // (3)
 $(pa(hd, \languagei{2}) \concatop \languagei{3}) \cup (tl \concatop (\languagei{2} \concatop \languagei{3}))$ == // (4)
 $pa(hd, \languagei{2} \concatop \languagei{3}) \cup (tl \concatop (\languagei{2} \concatop \languagei{3}))$ == // (5)
 $(hd::tl) \concatop (\languagei{2} \concatop \languagei{3})$
\end{ShortCode}
	
The explanation of each step is the following: 
\begin{enumerate}
	\item We apply Lemma~\ref{lem:clInductLeft}~and~\ref{lem:clContentEquals}. The first ensures that the two left hand sides are the same and the second ensures that the results of the combination are the same.
	\item We apply that concatenation is distributive over union, as stated in Lemma~\ref{lem:clLeftDistributiveAppend}.
	\item We apply the induction step, and apply the theorem for the right hand side.
	\item We apply the associativity in small. This step was proved in Lemma~\ref{lem:replaceConcatPrepend}.
	\item We apply Lemma~\ref{lem:clInductLeft} once more.
\end{enumerate} 
	
%	\item $((hd::tl) \concatop \languagei{2}) \concatop \languagei{3}$ ==
%	\item $(pa(hd, \languagei{2}) \cup (tl \concatop \languagei{2})) \concatop \languagei{3}$ ==
%	\item $(pa(hd, \languagei{2}) \concatop \languagei{3}) \cup ((tl \concatop \languagei{2}) \concatop \languagei{3})$ ==
%	\item $(pa(hd, \languagei{2}) \concatop \languagei{3}) \cup (tl \concatop (\languagei{2} \concatop \languagei{3}))$ ==
%	\item $(pa(hd, \languagei{2})  \concatop \languagei{3}) \cup (tl \concatop (\languagei{2} \concatop \languagei{3}))$ ==
%	\item $(hd:tl) \concatop (\languagei{2} \concatop \languagei{3})$

\subsubsection{Theorems about subset operation}

First, lets present some lemmas that are trivial for stainless.

\begin{lemma}[Reflexivity]
	\label{lem:sameAsSubset}
	Let $\languagei{1}, \languagei{2}$ be two languages such that $\languagei{1} == \languagei{2}$. In this case, we know that $\languagei{1} \subseteq \languagei{2}$.
\end{lemma}

\begin{lemma}[Transitivity]
	\label{lem:subsetOfTransitive}
	Let $\languagei{1}, \languagei{2}, \languagei{3}$ be three languages. If $\languagei{1} \subseteq \languagei{2}$ and  $\languagei{2} \subseteq \languagei{3}$ then  $\languagei{1} \subseteq \languagei{3}$.
\end{lemma}

Applying these, we can state two new lemmas:

\begin{lemma}
	\label{lem:sameAsSubsetTrans}
	Let $\languagei{1}, \languagei{2}, \languagei{3}$ be three languages. If $\languagei{1} \subseteq \languagei{2}$ and  $\languagei{2} == \languagei{3}$ then  $\languagei{1} \subseteq \languagei{3}$.
\end{lemma}

\begin{lemma}
	\label{lem:sameAsSubsetTrans2}
	Let $\languagei{1}, \languagei{2}, \languagei{3}$ be three languages. If $\languagei{1} == \languagei{2}$ and  $\languagei{2} \subseteq \languagei{3}$ then  $\languagei{1} \subseteq \languagei{3}$.
\end{lemma}

We can use the subset operation to prove equivalence of languages.
\begin{lemma}[Subset-Equivalence]
	\label{subsetSupersetSame}
	For all languages $\languagei{1}, \languagei{2}$ such that $\languagei{1} \subseteq \languagei{2}$ and $\languagei{2} \subseteq \languagei{1}$ we have $\languagei{1} == \languagei{2}$
\end{lemma}

We can also state, that if a language is subset to an other, than there has to be a language such that if we take the union of that and the smaller language, we get the bigger one.

\begin{lemma}
	\label{lem:subsetSplit}
	For all languages $\languagei{1}, \languagei{2}$ such that $\languagei{1} \subseteq \languagei{2}$ there is a language $\languagei{3}$ such that $(\languagei{1} \cup \languagei{3}) == \languagei{2}$.
\end{lemma}

However, instead of using the existential quantifier, it would be easier to simply specify $\languagei{3}$ as $\languagei{2} \setminus \languagei{1}$.

We can also state some subset lemmas with the union operation.

\begin{lemma}
	\label{lem:inUnionSubset}
	For all languages $\languagei{1}, \languagei{2}$ we have $\languagei{1} \subseteq (\languagei{1} \cup \languagei{2})$ and $\languagei{2} \subseteq (\languagei{1} \cup \languagei{2})$
\end{lemma}

We can also say that if two languages are separately subset of an other, their union will also be subset of that language.

\begin{lemma}
	\label{lem:unionSubset}
	Let $\languagei{1}, \languagei{2}, \languagei{3}$ be three languages. If $\languagei{1} \subseteq \languagei{3}$ and $\languagei{2} \subseteq \languagei{3}$ then $(\languagei{1} \cup \languagei{2})\subseteq \languagei{3}$
\end{lemma}

Based on these, we can state the following lemmas.

\begin{lemma}
	\label{lem:concatSubset}
	Let $\languagei{1}, \languagei{2}, \languagei{3}$ be three languages. If $\languagei{1} \subseteq \languagei{2}$ then $\languagei{1} \concatop \languagei{3} \subseteq \languagei{2} \concatop \languagei{3}$.
\end{lemma}

Stainless can not prove this lemma on its own, but we can combine previous lemmas to show that it actually holds.

\begin{ShortCode}{Stainless}
 $\languagei{1} \concatop \languagei{3} \subseteq $ //(1)
 $(\languagei{1}  \concatop \languagei{3}) \cup ((\languagei{2} \setminus \languagei{1}) \concatop \languagei{3}) $ == //(2)
 $(\languagei{1} \cup (\languagei{2} \setminus \languagei{1})) \concatop \languagei{3} $        == //(3)
 $\languagei{2} \concatop \languagei{3}$
\end{ShortCode}

In step 1) we applied Lemma~\ref{lem:inUnionSubset}, in step 2) we used distributivity as stated in Lemma~\ref{lem:clRightDistributiveAppend} and in step 3) we applied Lemma~\ref{lem:subsetSplit}. Eventually we have to apply Lemma~\ref{lem:sameAsSubsetTrans} to show that subset relation is transitive through equality.

The following lemma can be proved similarly:
\begin{lemma}
	\label{lem:concatSubset2}
	Let $\languagei{1}, \languagei{2}, \languagei{3}$ be three languages. If $\languagei{2} \subseteq \languagei{3}$ then $\languagei{1} \concatop \languagei{2} \subseteq \languagei{1} \concatop \languagei{3}$.
\end{lemma}

The proofs for the previous two lemmas are presented in functions \inline{concatSubset} and \inline{concatSubset2}.

\subsubsection{Theorems about the power of languages}

We defined the power operation in two different ways, one that unfolds to the left and one that unfolds to the right. Having proved associativity, we feel that eventually, these two definition are the same.


\begin{theorem}[Power definition equality]
	\label{the:couldHaveDefinedOtherWay}
	For all language $\languagesymbol$ and $i \in \nats$ we have $\languagesymbol \text{\textasciicircum~} i == \languagesymbol \text{\textasciicircum:~} i$
\end{theorem}

To prove this theorem we have to separate three cases.

If $i$ is $0$, by definition both sides will be $\unitlang$.

If $i$ is $1$ then 
\begin{ShortCode}{Stainless}
 $\languagesymbol$^1 == //by definition
 $\languagesymbol \concatop (\languagesymbol$^0) == //by definition
 $\languagesymbol \concatop \unitlang$ == // (1)
 $\languagesymbol$ == //(2)
 $\unitlang \concatop \languagesymbol$ == //by definition
 $(\languagesymbol$:^0$) \concatop \languagesymbol$ == //by definition
 $\languagesymbol$:^1
\end{ShortCode}

In step 1) we applied Theorem~\ref{the:rightUnitConcat}, and in step 2) we applied Theorem~\ref{the:leftUnitConcat}.

Otherwise, we can apply induction. We know that $i \ge 2$.

\begin{ShortCode}{Stainless}
 $\languagesymbol \concatop (\languagesymbol$^i-1) == // (1) 
 $\languagesymbol \concatop (\languagesymbol$:^i-1) == //(2)
 $\languagesymbol \concatop ((\languagesymbol$:^i-2)$\concatop \languagesymbol$) == //(3)
 $(\languagesymbol \concatop (\languagesymbol$:^i-2))$\concatop \languagesymbol$ == //(4)
 $(\languagesymbol \concatop (\languagesymbol$^i-2))$\concatop \languagesymbol$ == //(5)
 $(\languagesymbol$^i-1) $\concatop \languagesymbol$ == //(6)
 $(\languagesymbol$:^i-1$) \concatop \languagesymbol$ ==  //(7)
 $\languagesymbol$:^1
\end{ShortCode}

During the proof, we applied induction three times, in steps (1), (4) and (6). We used associativity (Theorem~\ref{the:associativity}) in step (3). In all other steps the proof only uses the definition and the lemmas about combining equivalent languages (Lemmas~\ref{lem:clContentEquals}~and~\ref{lem:clContentEquals2}). Note that the termination is alway ensured as we always in each induction step, the value of $i$ decreases. Furthermore, since in this case we know that $i \ge 2$, in the recursive calls $i \ge 0$ is ensured.

For the implementation of the proof, see \inline{couldHaveDefinedOtherWay}.

From now on, lets simply refer to both as $\languagesymbol^i$.

We can see that any language to the power of zero is a unit language just like in the case of the power of real numbers. But we can state more lemmas similar to that:

\begin{lemma}[First power of languages]
	\label{lem:langToFirst}
	For any language $\languagesymbol$ we have $\languagesymbol^1 == \languagesymbol$.
\end{lemma}

The proof for this lemma can be seen in function \inline{langToFirst} and it is really similar to the $i=1$ case in Theorem~\ref{the:couldHaveDefinedOtherWay}.

\begin{lemma}[Power of unit language]
	\label{lem:unitLangPow}
	For any $i \in \nats$ we have $\unitlang^i == \unitlang$
\end{lemma}

The proof for this can be shown using induction and is presented in function \inline{unitLangPow}.

An other useful lemma would be to show that sums in the exponent can be expanded into concatenations. However, for now it is enough to prove something weaker, for only two languages.

\begin{lemma}[Language to the sum]
	\label{lem:powSum}
	For any language $\languagesymbol$ and numbers $a,b \in \nats$ we have $\languagesymbol ^ {a+b} == (\languagesymbol ^ b) \concatop (\languagesymbol ^ b)$.
\end{lemma}

The proof for this lemma can be shown applying induction and associativity, as it is implemented in function \inline{powSum}.

\subsubsection{Theorems About the Close of Languages}

We defined the close operator the get closer to the notion of star. With this implementation, we can not implement infinite languages, but we can say that a word $\wordsymbol \in \languagepow{*}$ iff. $\exists i \in \nats. \wordsymbol \in \languageclose{i}$ 

We can also state some lemmas concerning the close of empty and unit languages.

\begin{lemma}[Close of Empty Language]
	\label{lem:nullLangClose}
	For every $i \in \nats$ we have $\close{\emptyset}{i} == \unitlang$
\end{lemma}

\begin{lemma}[Close of Unit Language]
	\label{lem:unitLangClose}
	For every $i \in \nats$ we have $\close{\unitlang}{i} == \unitlang$
\end{lemma}

These two can be proved easily using induction, as shown in the code at functions \inline{nullLangClose} and \inline{unitLangClose}.

A useful property of close that $\languageclose{i} \subseteq \languageclose{j}$ is equivalent to $i \le j$.

\begin{lemma}
	\label{lem:subsetCloseLe}
	$\forall \languagesymbol.$  $\languageclose{i} \subseteq \languageclose{j}$ iff. $i \le j$.
\end{lemma}

This can also be proved inductively, as shown in function \inline{subsetCloseLe}.

Finally, we would like to define something like Lemma~\ref{lem:powSum} for the close operation.

\begin{lemma}
	\label{lem:sumCloseSame}
	For every language $\languagei{}$ and $a,b \in \nats$ we have $\languageclose{a+b} == \languageclose{a} \concatop \languageclose{b} $
\end{lemma}

However, proof of this would be really difficult, so instead, lets just use a weaker form of this lemma.

\begin{lemma}
	\label{lem:sumClose}
	For every language $\languagei{}$ and $a,b \in \nats$ we have $\languageclose{a} \concatop \languageclose{b} \subseteq \languageclose{a+b}$
\end{lemma}

For the proof of the latter see function \inline{sumClose} in the code.

\subsection{Theorems About Regular Expressions}
\label{sect:thmRegEx}
We would like to show that for our regular expression implementation (as defined in Section~\ref{sect:regExImpl}) some properties hold. Recall that the implementation only allows two languages, $\languagei{1}$ and $\languagei{2}$. We would like to prove that $\forall \regex \in \regexsetl.$ if a language $\languagei{}$ is defined by $\regex$ then $\languagei{} \in \pow{(\languagei{1} \cup \languagei{2})}{*}$. Eliminate the star from that expression!

\begin{theorem}
	For every regular expression $\regex$ defined over the languages $\languagei{1}, \languagei{2}$, if $\languagei{}$ is defined by r, $\exists i \in \nats. ~ \languagesymbol \subseteq \close{(\languagei{1} \cup \languagei{2})}{i}$
\end{theorem}

In order to prove this, for each regular expression we have to guess the exponent $i$. For this, we have to define an other method \inline{evalExp():BigInt}. The value of it:

\begin{itemize}
	\item for \inline{L1} and \inline{L2} is $1$,
	\item for \inline{Union(r1,r2)} is \inline{max(r1.evalExp(), r2.evalExp())},
	\item for \inline{Conc(r1,r2)} is  \inline{r1.evalExp() + r2.evalExp()}
	\item for \inline{Pow(r,n)} is  \inline{r.evalExp() * n}
\end{itemize}

We can state a lemma nearly equivalent to the previous theorem:
\begin{lemma}
	\label{lem:regexSubsetStar}
	For every regular expression $\regex$ defined over the languages $\languagei{1}, \languagei{2}$, if $\languagei{}$ is defined by r, if i = \inline{r.evalExp()} then $\languagesymbol \subseteq \close{(\languagei{1} \cup \languagei{2})}{i}$
\end{lemma}

For \inline{L1} and \inline{L2} it is trivial that $i=1$ is sufficient because of Lemma~\ref{lem:inUnionSubset}.

For \inline{Union(r1,r2)} we can say that $\close{(\languagei{1} \cup \languagei{2})}{a} \subseteq \close{(\languagei{1} \cup \languagei{2})}{max(a,b)}$ and $\close{(\languagei{1} \cup \languagei{2})}{b} \subseteq \close{(\languagei{1} \cup \languagei{2})}{a,b}$ (a,b are \inline{r1.evalExp()} and \inline{r2.evalExp()}). From there we can apply the subset distributivity over union (Lemma~\ref{lem:unionSubset}) and with induction and subset transitivity (Lemma~\ref{lem:subsetOfTransitive}) we can prove the statement.

For \inline{Concat(r1,r2)} we chose \inline{evalExp = r1.evalExp() + r2.evalExp()}. Here the proof is similar to the previous case, but now we use Lemma~\ref{lem:sumClose} about the union of closes being contained in the close of sum. Since we only want to prove inclusion, the weaker (and proved) for of the lemma is sufficient.

We can say that \inline{Pow(r,n)} can be expressed as \\ \inline{Conc(r, Conc(r, Conc(..., Conc(r, Pow(r,1)) ...)))} which is \\ \inline{Conc(r, Conc(r, Conc(..., Conc(r, r) ...)))}, so this is only a syntactic sugar in regular expressions. For this reason, lets omit this case in the proofs.

Even though the proof is sound in paper, its Stainless implementation is not yet implemented completely, some steps are not verifying. The initial version of the proof is included in function \inline{regexSubsetStar} in the file \inline{RegEx.scala}.

For the previous proof to be correct, we have to show that while taking the close of a language we do not perform an invalid operation, namely each exponent is nonnegative.

\begin{lemma}
	\label{lem:evalExpPositive}
	For each regular expression $\regex$ we have \inline{r.evalExp() > 0}.
\end{lemma}

The proof for this lemma is trivial as for the constant values the exponent is positive, and both maximum selection and addition (of positive numbers) preserves positiveness.
